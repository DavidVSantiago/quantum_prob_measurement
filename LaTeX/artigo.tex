\documentclass[12pt, a4paper]{article}
\usepackage[utf8]{inputenc}
\usepackage[brazil]{babel} % Opcional: para texto em português
\usepackage{amsmath} % Para equações matemáticas
\usepackage{braket} % Para notação de Dirac (|ψ⟩)
\usepackage{hyperref} % Para links clicáveis (opcional)
\usepackage{graphicx} % Para inserir quadros e diagramas
\usepackage{booktabs} % Para tabelas profissionais
\usepackage{lipsum}

\title{Medição de Probabilidade de Estados Quânticos em Qubits Únicos}
\author{David Santiago}
\date{\today}

\begin{document}

\maketitle

\begin{abstract}
Texto a ser colocado posteriormente
\end{abstract}

\keywords{computação quântica, estados quânticos, qubits. algoritmos}

\section{Introdução}
\label{sec:intro}
Texto a ser colocado posteriormente

\section{Definição matemática}
\label{sec:definicao_mate}

Nesta seção serão apresentados os modelos matemáticos utilizados para a implementação do algoritmo da medição de probabilidade dos estados quânticos. É importante salientar que esses modelos já estão bem estabelecidos pela literatura da computação quântica, com destaque para Nielsen e Chuang\cite{nielsen2010}.

O objetivo desta seção é detalhar o processo de transformações algébricas aplicadas aos modelos, no âmbito da obtenção de um modelo de menor complexidade numérica, que elimina a necessidade de operações com raízes e permite a implementação de um algoritmo clássico computacionalmente mais eficiente, conforme demonstrado nos resultados dos experimentos (seção \ref{sec:resultados}).

\subsection{Representação do estado quântico de bit único \label{sec:rep_estados_quanticos}}

Um estado quântico de um qubit único é representado pela notação de Dirac \(\ket{\psi}\), conforme é ilustrado na equação \eqref{eq:estado_qubit}:

\begin{equation}
\ket{\psi} = \alpha\ket{0} + \beta\ket{1}
\label{eq:estado_qubit}
\end{equation}

onde as simbologias \(\alpha\) e \(\beta\) são números complexos que definem as amplitudes de probabilidades, de tal forma que \(|\alpha|^2\) determina a probabilidade de obtenção do qubit \(\ket{0}\) e \(|\beta|^2\) determina a probabilidade de obtenção do qubit \(\ket{1}\).

Estados quânticos conhecidos, tais como os de Bell e de Hadamard, são tipicamente representados por frações envolvendo raízes quadradas \cite{nielsen2010, preskill1998}. Por exemplo, o estado de Hadamard é definido como:

\begin{equation}
\ket{+} = \frac{1}{\sqrt{2}}(\ket{0} + \ket{1})
\label{eq:hadamard}
\end{equation}

Neste trabalho, propomos uma representação alternativa para \(\alpha\) e \(\beta\), expressando-os como frações de números complexos com denominadores reais:

\begin{equation}
\alpha = \frac{x_1 + y_1 i}{d_1},
\label{eq:alpha}
\end{equation}

\begin{equation}
\beta = \frac{x_2 + y_2 i}{d_2}
\label{eq:beta}
\end{equation}

onde:

\begin{itemize}
\item \(x_1, y_1, x_2, y_2 \in \mathbb{R}\) são as partes real e imaginária dos numeradores,
\item \(d_1, d_2 \in \mathbb{R}\) são denominadores que garantem a normalização do estado (\(|\alpha|^2 + |\beta|^2 = 1\)).
\end{itemize}

Essa representação elimina a manipulação direta de raízes quadradas durante o cálculo de probabilidades.

\subsection*{Exemplo prático}

No estado \(\ket{+}\), \(\sqrt{2}\) é tratado como símbolo, evitando cálculos explícitos de raízes:

\begin{align*}
\alpha = \frac{1}{d}, \quad \beta = \frac{1}{d}, \quad \text{com } d = \sqrt{2}
\end{align*}

Isso permite que operações como \(|\alpha|^2=\frac{1}{d^2}\) sejam realizadas sem calcular \(\sqrt{2}\) explicitamente, simplificando a implementação computacional.

\subsection{Cálculo dos valores de probabilidade}

Um bom ponto de partida para a medição das probabilidades é calcular diretamente os valores de \(|\alpha|^2\) e \(|\beta|^2\). Dado que \(\alpha\) e \(\beta\) são números complexos na forma \(\frac{x+yi}{d}\) e dado que \(|\alpha|\) e \(|\beta|\) são os módulos do fatores de amplitude, tem-se que:

\begin{equation}
|\alpha| = \sqrt{\left(\frac{x_1}{d_1}\right)^2+\left(\frac{y_1}{d_1}\right)^2} =  \sqrt{\frac{x_1^2+y_1^2}{d_1^2}},
\label{eq:mod_alpha}
\end{equation}

\begin{equation}
|\beta| = \sqrt{\left(\frac{x_2}{d_2}\right)^2+\left(\frac{y_2}{d_2}\right)^2} =  \sqrt{\frac{x_2^2+y_2^2}{d_2^2}}
\label{eq:mod_beta}
\end{equation}

Como os valores de probabilidade de \(\ket{0}\) e \(\ket{1}\) são dados, respectivamente, por \(|\alpha|^2\) e \(|\beta|^2\), tem-se que:

\begin{equation}
|\alpha|^2 = \left(\sqrt{\frac{x_1^2+y_1^2}{d_1^2}}\right)^2=\frac{x_1^2+y_1^2}{d_1^2},
\label{eq:mod_alpha_sqr}
\end{equation}

\begin{equation}
|\beta|^2 = \left(\sqrt{\frac{x_2^2+y_2^2}{d_2^2}}\right)^2=\frac{x_2^2+y_2^2}{d_2^2}
\label{eq:mod_beta_sqr}
\end{equation}

Para simplificar operações, definimos:
\begin{itemize}
\item \(z \in \mathbb{Z}\) como uma representação simbólica para \(x^2+y^2\),
\item \(w \in \mathbb{Z}\) como uma representação simbólica para \(d^2\).
\end{itemize}

Assim:

\begin{equation}
|\alpha|^2 = \frac{z_1}{w_1},
\label{eq:mod_alpha_sq_final}
\end{equation}

\begin{equation}
|\beta|^2 = \frac{z_2}{w_2}
\label{eq:mod_beta_sq_final}
\end{equation}

Desta forma, tem-se \(\frac{z_1}{w_1}\) como a probabilidade de medir \(\ket{0}\) e \(\frac{z_2}{w_2}\) como a probabilidade de medir \(\ket{1}\).

\subsection*{Exemplo Numérico}

Para \(\alpha = \frac{1 + 3i}{\sqrt{10}}\):
\begin{table}[h]
\centering
\begin{tabular}{lc}
\toprule
\textbf{Passo} & \textbf{Cálculo} \\
\midrule
Tradicional & \(|\alpha|^2 = \left(\sqrt{\frac{1^2 + 3^2}{\sqrt{10}^2}}\right)^2 = 1\) \\
Proposto & \(z = 10\), \(w = 10\) \(\Rightarrow \frac{10}{10} = 1\) \\
\bottomrule
\end{tabular}
\end{table}

\begin{figure}[h]
\centering
\fbox{
\begin{minipage}{0.8\textwidth}
\textbf{Fluxo do cálculo do Algoritmo:}
\begin{enumerate}
\item Recebe \(\alpha = \frac{x + yi}{d}\)
\item Calcula \(z = x^2 + y^2\)
\item Calcula \(w = d^2\)
\item Retorna probabilidade \(\frac{z}{w}\)
\end{enumerate}
\end{minipage}
}
\caption{Processamento sem operações com raízes}
\label{fig:fluxo}
\end{figure}

\subsection{Cálculo de normalização}

Uma vez calculados os valores de probabilidade, é necessário verificar se estão normalizados antes de tê-los como resultado final da medição. Como citado na seção \ref{sec:rep_estados_quanticos}, o estado quântico está normalizado quando \(|\alpha|^2+|\beta|^2=1\).

A partir das equações (\ref{eq:mod_alpha_sq_final}) e (\ref{eq:mod_beta_sq_final}), definimos:

\begin{equation}
wf=\text{mmc}(w_1, w_2), \quad  \text{com } wf \in \mathbb{Z},
\label{eq:wf}
\end{equation}

\begin{equation}
zf=wf\left(\frac{z_1}{w_1}+\frac{z_2}{w_2}\right), \quad \text{com } zf \in \mathbb{Z}
\label{eq:zf}
\end{equation}

A escolha de \(wf=\text{mmc}(w_1, w_2)\) se deve à necessidade de um denominador comum, pois \(w_1\) e \(w_2\) podem assumir valores distintos.

Assim:

\begin{equation}
|\alpha|^2 + |\beta|^2 = \frac{z_1}{w_1} + \frac{z_2}{w_2} = \frac{zf}{wf}
\label{eq:zf_wf}
\end{equation}

A partir de então, tem-se que o estado está normalizado quando \(\frac{zf}{wf}=1\). Por outro lado, caso \(\frac{zf}{wf}\ne1\), é necessário realizar o processo de normalização, o qual consiste em recalcular novos valores para as amplitudes de probabilidade \(\alpha\) e \(\beta\). Este processo se inicia com a identificação do \textit{fator de normalização}, que é a raiz quadrada da soma dos quadrados dos módulos de \(\alpha\) e \(\beta\), dado pela expressão:

\begin{equation}
F=\sqrt{|\alpha|^2+|\beta|^2}=\sqrt{\frac{zf}{wf}}=\frac{\sqrt{zf}}{\sqrt{wf}}
\label{eq:norm_factor}
\end{equation}

Uma vez em posse desse fator, pode-se calcular os novos valores de amplitudes de probabilidade normalizados. Isso é feito dividindo valores de amplitude \(\alpha\) e \(\beta\) iniciais pelo fator de normalização. Desta forma, definimos:

\begin{equation}
\alpha'=\frac{\alpha}{F}=\frac{\frac{x_1+y_1i}{d_1}}{\frac{\sqrt{zf}}{\sqrt{wf}}}=\left(\frac{x_1+y_1i}{d_1}\right)\left(\frac{\sqrt{wf}}{\sqrt{zf}}\right)=\frac{\sqrt{wf}\left(x_1+y_1i\right)}{\sqrt{zf}\left(d_1\right)},
\label{eq:alpha_norm}
\end{equation}

\begin{equation}
\beta'=\frac{\beta}{F}=\frac{\frac{x_2+y_2i}{d_2}}{\frac{\sqrt{zf}}{\sqrt{wf}}}=\left(\frac{x_2+y_2i}{d_2}\right)\left(\frac{\sqrt{wf}}{\sqrt{zf}}\right)=\frac{\sqrt{wf}\left(x_2+y_2i\right)}{\sqrt{zf}\left(d_2\right)}
\label{eq:beta_norm}
\end{equation}

Uma vez normalizadas as amplitudes, conclui-se que \(|\alpha|^2\) representa a probabilidade de medir \(\ket{0}\) e \(|\beta|^2\) representa a probabilidade de medir \(\ket{1}\). Portanto, a partir das equações (\ref{eq:alpha_norm}) e (\ref{eq:beta_norm}), pode-se definir que:

\begin{equation}
|\alpha'|^2=\frac{\left(\sqrt{wf}\cdot x_1\right)^2+\left(\sqrt{wf}\cdot y_1\right)^2}{\left(\sqrt{zf}\cdot d_1\right)^2}=\frac{wf\cdot x_1^2+wf\cdot y_1^2}{zf\cdot d_1^2},
\label{eq:alpha_norm_sqr}
\end{equation}

\begin{equation}
|\beta'|^2=\frac{\left(\sqrt{wf}\cdot x_2\right)^2+\left(\sqrt{wf}\cdot y_2\right)^2}{\left(\sqrt{zf}\cdot d_2\right)^2}=\frac{wf\cdot x_2^2+wf\cdot y_2^2}{zf\cdot d_2^2}
\label{eq:beta_norm_sqr}
\end{equation}

Para simplificar operações, definimos:
\begin{itemize}
\item \(z' \in \mathbb{Z}\) como uma representação simbólica para \(wf\cdot x^2+wf\cdot y^2\),
\item \(w' \in \mathbb{Z}\) como uma representação simbólica para \(zf\cdot d^2\).
\end{itemize}

Assim:

\begin{equation}
|\alpha'|^2=\frac{z'_1}{w'_1},
\label{eq:alpha_norm_sqr_final}
\end{equation}

\begin{equation}
|\beta'|^2=\frac{z'_2}{w'_2}
\label{eq:beta_norm_sqr_final}
\end{equation}

Desta forma, tem-se \(\frac{z'_1}{w'_1}\) e \(\frac{z'_2}{w'_2}\) como a probabilidade de medir, respectivamente, \(\ket{0}\) e \(\ket{1}\) após o processo de normalização do estado quântico.

É importante ressaltar que estas transformações algébricas demonstradas diferem das simbologias e demonstrações mais formais encontradas na bibliografia. As transformações algébricas realizadas nas equações \ref{eq:alpha_norm_sqr} e \ref{eq:beta_norm_sqr} tiveram a intenção de resultar em uma expressão especialmente útil no contexto de eficiência computacional.
Na transformação:

\begin{align*}
\frac{\left(\sqrt{wf}\cdot x\right)^2+\left(\sqrt{wf}\cdot y\right)^2}{\left(\sqrt{zf}\cdot d\right)^2}=\frac{wf\cdot x^2+wf\cdot y^2}{zf\cdot d^2}
\end{align*}

observa-se a eliminação da necessidade de operações de extração de raízes (operações computacionalmente custosas).

\subsection*{Exemplo numérico completo}
Dado um estado quântico:

\begin{align*}
\ket{\psi}=\left(\frac{1+2i}{\sqrt{10}}\right)\ket{0}+\left(\frac{3-i}{\sqrt{10}}\right)\ket{1}
\end{align*}

o algoritmo proposto segue os passos da figura \ref{fig:ex_numerico}

\begin{figure}[h]
\centering
\fbox{
\begin{minipage}{0.8\textwidth}
\textbf{Fluxo do cálculo do Algoritmo:}
\begin{enumerate}
\item \(x_1=1,y_1=2,d_1=\sqrt{10},x_2=3,y_2=-1,d_2=\sqrt{10}\)
\item \(|\alpha|^2=\frac{x_1^2+y_1^2}{d_1^2}=\frac{1^2+2^2}{\sqrt{10}^2}=\frac{5}{10}=\frac{1}{2}\Rightarrow \frac{z_1}{w_1}=\frac{1}{2}\)
\item \(|\beta|^2=\frac{x_2^2+y_2^2}{d_2^2}=\frac{3^2+(-1)^2}
{\sqrt{10}^2}=\frac{10}{10}=1\Rightarrow \frac{z_2}{w_2}=\frac{1}{1}\)
\item \(\frac{zf}{wf}=\frac{z_1}{w_1}+\frac{z_2}{w_2}=\frac{1}{2}+1=\frac{3}{2}\Rightarrow \frac{zf}{wf} \neq1\), não normalizado!
\item \(F=\frac{\sqrt{zf}}{\sqrt{wf}}=\frac{\sqrt{3}}{\sqrt{2}}\), fator de normalização
\item \(\frac{z'_1}{w'_1}=\frac{wf\cdot x_1^2+wf\cdot y_1^2}{zf\cdot d_1^2}=\frac{2\cdot1^2+2\cdot2^2}{3\cdot\sqrt{10}^2}=\frac{2+8}{30}=\frac{1}{3}\)
\item \(\frac{z'_2}{w'_2}=\frac{wf\cdot x_2^2+wf\cdot y_2^2}{zf\cdot d_2^2}=\frac{2\cdot3^2+2\cdot1^2}{3\cdot\sqrt{10}^2}=\frac{18+2}{30}=\frac{2}{3}\)
\item prob. \(\ket{0}=33.\overline{3}\%\), prob. \(\ket{1}=66.\overline{6}\%\)
\end{enumerate}
\end{minipage}
}
\caption{Processamento sem operações com raízes}
\label{fig:ex_numerico}
\end{figure}

\section{algoritmo proposto}
\label{sec:algoritmo}
Texto a ser colocado posteriormente

\subsection{modelagem simbólica}
\label{sec:modelagem}
Texto a ser colocado posteriormente

\subsection{Implementação}
\label{sec:implementacao}
Texto a ser colocado posteriormente

\section{Testes e validação}
\label{sec:testes_validacao}
Texto a ser colocado posteriormente

\section{Resultados}
\label{sec:resultados}
Texto a ser colocado posteriormente

\section{Conclusões}
\label{sec:conclusoes}
Texto a ser colocado posteriormente

\begin{thebibliography}{9}
\bibitem{nielsen2010} Nielsen, M. A., \& Chuang, I. L. (2010). \textit{Quantum Computation and Quantum Information}. Cambridge University Press.
\bibitem{preskill1998} Preskill, J. (1998). \textit{Lecture Notes on Quantum Computation}. Caltech.
\end{thebibliography}

\end{document}