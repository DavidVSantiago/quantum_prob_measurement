\documentclass[12pt, a4paper]{article}
\usepackage[utf8]{inputenc}
\usepackage[brazil]{babel} % Opcional: para texto em português
\usepackage{amsmath} % Para equações matemáticas
\usepackage{braket} % Para notação de Dirac (|ψ⟩)
\usepackage{hyperref} % Para links clicáveis (opcional)

\title{Medição de Probabilidade de Estados Quânticos em Qubits Únicos}
\author{Seu Nome}
\date{\today}

\begin{document}

\maketitle

\section{Representação do Estado Quântico de Bit Único}
\label{sec:representacao}

Um estado quântico de um qubit único é representado pela notação de Dirac \(\ket{\psi}\) \cite{referencia1}, conforme ilustrado na Equação \eqref{eq:estado_qubit}:

\begin{equation}
\ket{\psi} = \alpha\ket{0} + \beta\ket{1},
\label{eq:estado_qubit}
\end{equation}

onde \(\alpha\) e \(\beta\) são números complexos denominados \textit{amplitudes de probabilidade}. As probabilidades de medir os estados \(\ket{0}\) e \(\ket{1}\) são dadas por \(|\alpha|^2\) e \(|\beta|^2\), respectivamente.

Estados quânticos conhecidos, como os estados de Bell e de Hadamard, são tipicamente representados por frações envolvendo raízes quadradas \cite{nielsen2010, preskill1998}. Por exemplo, o estado de Hadamard é definido como:

\begin{equation}
\ket{+} = \frac{1}{\sqrt{2}}(\ket{0} + \ket{1}).
\label{eq:hadamard}
\end{equation}

Neste trabalho, propomos uma representação alternativa para \(\alpha\) e \(\beta\), expressando-os como frações de números complexos com denominadores reais:

\begin{align}
\alpha &= \frac{x_1 + \gamma i}{d_1}, \label{eq:alpha} \\
\beta &= \frac{x_2 + \delta i}{d_2}, \label{eq:beta}
\end{align}

onde:
\begin{itemize}
\item \(x_1, \gamma, x_2, \delta \in \mathbb{R}\) são as partes real e imaginária dos numeradores,
\item \(d_1, d_2 \in \mathbb{R}\) são denominadores que garantem a normalização do estado (\(|\alpha|^2 + |\beta|^2 = 1\)).
\end{itemize}

Essa representação elimina a manipulação direta de raízes quadradas durante o cálculo de probabilidades. Por exemplo, o estado \(\ket{+}\) pode ser reescrito como:

\begin{equation}
\alpha = \frac{1}{d}, \quad \beta = \frac{1}{d}, \quad \text{com } d = \sqrt{2}.
\label{eq:exemplo}
\end{equation}

\section*{Referências}
\begin{thebibliography}{9}
\bibitem{referencia1} Autor. \textit{Título do Artigo}. Revista, Ano.
\bibitem{nielsen2010} Nielsen, M. A., \& Chuang, I. L. (2010). \textit{Quantum Computation and Quantum Information}. Cambridge University Press.
\bibitem{preskill1998} Preskill, J. (1998). \textit{Lecture Notes on Quantum Computation}. Caltech.
\end{thebibliography}

\end{document}